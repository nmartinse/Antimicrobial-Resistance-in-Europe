\documentclass[../main.tex]{subfiles}

\begin{document}

En esta sección se van a hacer dos posibles consultas que harían distintos usuarios a la base de datos, para lo que se usan las queries o consultas. Una consulta sirve encontrar y extraer elementos y atributos de una base de datos.

Para realizar estas consultas se va a usar XQuery, un lenguaje para realizar consultas en documentos XML. Se basa en funciones, expresiones XPath y predicados.

Las operaciones más comunes para hacer queries son SELECT, que selecciona los datos que serán visibles en la tabla resultado; FROM, para señalar la o las tablas de las cuales quiero obtener la información, WHERE, utilizada para delimitar la información que queremos obtener.

\subsection{Query 1}


\subsection{Query 2}

\end{document}