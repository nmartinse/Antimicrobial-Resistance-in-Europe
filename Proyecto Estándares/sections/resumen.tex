\documentclass[../main.tex]{subfiles}

\begin{document}
%% begin abstract format
\makeatletter
\renewenvironment{abstract}{%
    \if@twocolumn
      \section*{Resumen \\}%
    \else %% <- here I've removed \small
    \begin{flushright}
        {\filleft\Huge\bfseries\fontsize{48pt}{12}\selectfont Resumen\vspace{\z@}}%  %% <- here I've added the format
        \end{flushright}
      \quotation
    \fi}
    {\if@twocolumn\else\endquotation\fi}
\makeatother
%% end abstract format
%% begin abstract format
\makeatletter
\renewenvironment{abstract}{%
    \if@twocolumn
      \section*{Resumen \\}%
    \else %% <- here I've removed \small
    \begin{flushright}
        {\filleft\Huge\bfseries\fontsize{48pt}{12}\selectfont Resumen\vspace{\z@}}%  %% <- here I've added the format
        \end{flushright}
      \quotation
    \fi}
    {\if@twocolumn\else\endquotation\fi}
\makeatother
%% end abstract format
\begin{abstract}

En este proyecto se va a desarrollar una página web sobre bacterias que han desarrollado resistencia a los antibióticos en Europa.

Los datos se van a extraer del siguiente enlace: \href{https://www.kaggle.com/datasets/samfenske/euro-resistance}{kaggle.com/datasets/euro-resistance}. 

Se incluyen las siguientes columnas:

\begin{itemize}
    \item Distribución: se refiere a si los datos se extrajeron de un género o grupo de edad en particular.
    \item RegionName: se refiere al país de la institución informante.
    \item Tiempo: es el año.
    \item  Categoría: se refiere al grupo de edad o al género (dependiendo de la distribución)
    \item  Valor: se refiere al porcentaje de bacterias que fueron resistentes al grupo de antibióticos
    \item Bacterias: se refiere a las bacterias en las que se estudió la resistencia.
    \item Antibiótico: se refiere al grupo de antibióticos que se usó para matar la bacteria.
\end{itemize}


\bfseries{\large{Palabras clave:}} Bacteria, Antibiótico, C

\end{abstract}
\end{document}


