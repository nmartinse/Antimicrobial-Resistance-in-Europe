\documentclass[../main.tex]{subfiles}

\begin{document}
%% begin abstract format
\makeatletter
\renewenvironment{abstract}{%
    \if@twocolumn
      \section*{Resumen \\}%
    \else %% <- here I've removed \small
    \begin{flushright}
        {\filleft\Huge\bfseries\fontsize{48pt}{12}\selectfont Resumen\vspace{\z@}}%  %% <- here I've added the format
        \end{flushright}
      \quotation
    \fi}
    {\if@twocolumn\else\endquotation\fi}
\makeatother
%% end abstract format
%% begin abstract format
\makeatletter
\renewenvironment{abstract}{%
    \if@twocolumn
      \section*{Resumen \\}%
    \else %% <- here I've removed \small
    \begin{flushright}
        {\filleft\Huge\bfseries\fontsize{48pt}{12}\selectfont Resumen\vspace{\z@}}%  %% <- here I've added the format
        \end{flushright}
      \quotation
    \fi}
    {\if@twocolumn\else\endquotation\fi}
\makeatother
%% end abstract format
\begin{abstract}

En este proyecto se va a desarrollar una página web que incluya información sobre bacterias con resistencia a los antibióticos en Europa.

\hfill


Se va a comenzar por extraer información de un dataset (en formato XML). A este archivo se le aplicarán transformaciones XSLT para poder integrar la información en una página web. Esta página web será diseñada para que los distintos usuarios puedan visualizar de forma sencillos los datos recogidos. Para ello se van a crear dos interfaces, una dirigida a investigadores y otra dirigida a usuarios que simplemente quieran informarse sobre el tema.

\hfill


Además, se realizarán posibles consultas que puedan realizar los usuarios sobre el dataset en lenguajes como XQuery y SPARQL.

\hfill


El proyecto se aloja en el siguiente repositorio de gitHub: 

\href{https://github.com/nmartinse/Antimicrobial-Resistance-in-Europe}{github.com/Antimicrobial-Resistance-in-Europe}.  

\hfill

Y la implementacion de la páguina web final está hecha en GitHub Pages, y se puede acceder con el siguiente enlace:

\href{https://Archerd6.github.io/Proyecto_Estandares_de_Datos_Web}{Archerd6.github.io/Proyecto Estandares de Datos Web}.

\hfill

[a desarrollar conforme se vaya realizando el proyecto]

\bfseries{\large{Palabras clave:}} Bacteria, Antibiótico, Base de datos, Página web, HTML, XSLT, XML

\end{abstract}
\end{document}


