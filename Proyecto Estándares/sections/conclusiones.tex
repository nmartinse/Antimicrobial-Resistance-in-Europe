\documentclass[../main.tex]{subfiles}

\begin{document}



La página web creada facilita la búsqueda de información sobre bacterias con resistencia a antibióticos, para los dos grupos potenciales de usuarios (personas no especializadas e investigadores).

Las tecnologías de estandarización de datos han sido de gran utilidad para el desarrollo de este trabajo. Es gracias a la familia de lenguajes XML que se ha podido procesar fácilmente el dataset en formato tabular que se obtuvo del repositorio público. 

Se han obtenido transformaciones que muestran los datos necesarios para cada una de las vistas gracias a los scripts xslt, que en formato de salida html han sido añadidas al código de cada una de las interfaces de la web.

Además, gracias a esta familia de lenguajes se han podido realizar distintas consultas xquery haciendo uso del lenguaje xpath.

\hfill

Finalmente, los estandares de RDF y OWL han sido de utilidad para el planteamiento de un modelo de ralaciones semanticas, sobre el que se han realizado consultas SparQL 

\hfill



\end{document}